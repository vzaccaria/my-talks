% Created 2018-01-30 Tue 16:48
% Intended LaTeX compiler: pdflatex
\documentclass[aspectratio=169]{beamer}
\usepackage[utf8]{inputenc}
\usepackage[T1]{fontenc}
\usepackage{graphicx}
\usepackage{grffile}
\usepackage{longtable}
\usepackage{wrapfig}
\usepackage{rotating}
\usepackage[normalem]{ulem}
\usepackage{amsmath}
\usepackage{textcomp}
\usepackage{amssymb}
\usepackage{capt-of}
\usepackage{hyperref}
\usetheme{metropolis}
\usepackage{fixltx2e}
\usepackage{graphicx}
\usepackage{longtable}
\usepackage{float}
\usepackage{wrapfig}
\usepackage{rotating}
\usepackage[normalem]{ulem}
\usepackage{amsmath}
\usepackage{textcomp}
\usepackage{marvosym}
\usepackage{wasysym}
\usepackage{amssymb}
\usepackage{hyperref}
\usepackage{xcolor}
\usepackage{stmaryrd}
\usepackage{fontawesome}
\usepackage{shellesc}
\tolerance=1000
\usepackage{minted}
\usepackage{ifplatform}
\setmonofont[Scale=0.75]{FiraCode-Regular}
\ifmacosx
\newfontfamily\myunicodefallback{Menlo}
\newfontfamily\myunicodefallbackb{DejaVu Sans}
\fi
\iflinux
\newfontfamily\myunicodefallback{Menlo-Regular}
\newfontfamily\myunicodefallbackb{DejaVu Sans}
\fi
\newcommand{\uc}[1]{{\myunicodefallbackb #1}}
\definecolor{links}{HTML}{0086F7}
\hypersetup{colorlinks,linkcolor=,urlcolor=links}
\newcommand{\rpr}[1]{{\llbracket \texttt{#1} \rrbracket}}
\newcommand{\rprtwo}[1]{${\llbracket \texttt{#1} \rrbracket}_2$}
\newcommand{\rprten}[1]{${\llbracket \texttt{#1} \rrbracket}_{10}$}
\newcommand{\rprctwo}[1]{${\llbracket \texttt{#1} \rrbracket}_{C2}$}
\usepackage{fancyvrb}
\usepackage{setspace}
\usepackage{etoolbox}
\appto\verbatim{\tiny\myunicodefallback\baselineskip=-\maxdimen\lineskip=0pt}
\usepackage{semantic}
\setmonofont[
Contextuals={Alternate}, Scale=0.7,
Scale=MatchLowercase % (LD.)
]{Fira Code}
\newcommand{\semc}[1]{\mathcal{C}|[#1|]}
\newcommand{\semcd}[2]{\mathcal{C}|[#1|] & = & #2}
\newcommand{\seme}[1]{\mathcal{E}|[#1|]}
\newcommand{\semed}[2]{\mathcal{E}|[#1|] & = & #2}
\newcommand{\tme}{\mathbb{N}}
\newcommand{\reals}{\mathbb{R}}
\newcommand{\envvars}{|[\Gamma|]}
\newcommand{\describe}[2]{\underbrace{#1}_{\text{#2}}}
\newcommand{\describeu}[2]{\overbrace{#1}^{\text{#2}}}
\newcommand{\assetexchange}[3]{\text{#1}(#2 \rightarrow #3)}
\newcommand{\observe}[2]{\text{Obs}(#1,#2)}
\newcommand{\event}[1]{\text{#1}}
\newcommand{\ite}[4]{\text{if}~{#1}~\text{in}~{#2}~\text{then}~{#3}~\text{else}~#4}
\usetheme{default}
\author{Vittorio Zaccaria}
\date{\today}
\title{DSLs in finance, an overview}
\let\oldhref\href
\def\href#1#2{\oldhref{#1}{\uline{#2}}}
\let\oldfootnote\footnote
\renewcommand\footnote[1][]{\oldfootnote[frame,#1]}
\setbeamerfont{caption}{size=\scriptsize}
\hypersetup{
 pdfauthor={Vittorio Zaccaria},
 pdftitle={DSLs in finance, an overview},
 pdfkeywords={},
 pdfsubject={},
 pdfcreator={Emacs 25.3.1 (Org mode N/A-fixup)}, 
 pdflang={English}}
\begin{document}

\maketitle
\begin{frame}{Outline}
\tableofcontents
\end{frame}




\section{Introduction}
\label{sec:org644c224}
\section{Contract languages}
\label{sec:orgeafcb26}
\begin{frame}[fragile,label={sec:org5e8f9be}]{Certified symbolic management cite:Bahr:2015hm}
 \begin{itemize}
\item Contracts are expressed in relative times

\item Can describe stipulation between multiple parties (\texttt{\&} operator)

\item Can describe \emph{observable external decisions}:

$$\ite{\describe{\observe{\event{X exercises option}}{0}}{eval'd at the beginning of each day}}{\describeu{90}{days' range}}{100*\describe{\assetexchange{DKK}{Y}{X}}{atomic exchange contract}}{0}$$
\end{itemize}
\end{frame}

\begin{frame}[label={sec:org883d622}]{Findel cite:Biryukov:2017ip}
\begin{itemize}
\item Absolute times;

\item Syntax similar to others but no intuitive reference to parties involved as
contracts change over time:

$$\describe{Give(10 * One(USD))}{Bob pays 10 now} \wedge \describeu{At(now+1~years, 11 * One(USD))}{Bob expects to receive 11 in a year}$$
\end{itemize}
\end{frame}




\section{Semantics}
\label{sec:org1f2e27a}
\begin{frame}[label={sec:org37d5664}]{Certified symbolic management cite:Bahr:2015hm}
Subdivides semantics into \alert{contract} and \alert{expression} (denotational) semantics.
Contract semantics maps expressions into a cash-flow trace:
$$ C: \envvars \rightarrow \describeu{\tme \rightarrow \describe{Party \times Party \times Asset \rightarrow \reals}{transaction}}{trace} $$

for example (note the delay \(\uparrow\) and \emph{unit transfer} \(\rightarrow\) operators)\footnote{\(\delta\) = Kronecker's delta}:
\begin{equation}
\begin{array}{rcl}
\semcd{0}{\lambda n. \lambda t . 0}\\
\semcd{c1 \& c2}{\semc{c1} + \semc{c2}} \\
\semcd{d \uparrow c}{\lambda n. \semc{c}(n - d)} \\
\semcd{a(p_1 \rightarrow p_2)}{\lambda n.\lambda t.\delta_{0,(p_1,p_2,a)}(n,t) - \delta_{0,(p_2,p_1,a)}(n,t)} \\
\end{array}
\end{equation}
\end{frame}
\begin{frame}[label={sec:org2d05a0f}]{Certified symbolic management}
Contract transforms consist in specialisation and advancement, i.e.,
instantiation of a contract to a concrete starting time or simplification.
Consider the following contract:
  $$\assetexchange{DKK}{Y}{Z} ~\&~ \ite{\observe{\event{X defaults}}{0}}{30}{\assetexchange{DKK}{Z}{Y}}{0}$$

and assume that $$default(X,i)=\top~if~i=15,~\bot~otherwise$$ Then, at time
\(i=16\), the contract can be transformed into:
  $$\assetexchange{DKK}{Y}{Z} ~\&~ \assetexchange{DKK}{Z}{Y} \sim 0$$
\end{frame}



\section{Software verification and certified software}
\label{sec:org3afb791}
\section{Type systems}
\label{sec:org93fc752}
\begin{frame}[label={sec:orgcf699c4}]{Certified symbolic management}
\begin{itemize}
\item \alert{Problem}: Simple expressions could involve non-causality, e.g.:

$$\describe{\textbf{obs}(FX(USD, DKK), 1)}{tomorrow's observation} \times \describe{DKK(X \rightarrow Y)}{pay today}$$

\item \alert{Solution}: time-indexed types;
\end{itemize}
\end{frame}

\begin{frame}[label={sec:org86187bf}]{Certified symbolic management}
\alert{Examples} of typing rules using time-indexed types:

\begin{itemize}
\item an observation at time \(t\) is available at all times \(t'\) after \(t\):
$$\inference{t \leq t'}{\Gamma |- \textbf{Obs}(l,t) : \tau^{t'}}$$

\item an expression \(e\) can only meaningfully scale a contract \(c\) if \(e\) is
available at some time \(t'\) and \(c\) makes no stipulations strictly before
\(t'\): $$\inference{\Gamma |- e: Real^{t'} & \Gamma |- c: Contr^{t'} & t \leq t'}{\Gamma |- e \times c:
  Contr^t}$$
\end{itemize}
\end{frame}

\begin{frame}[label={sec:org3604ee6}]{Functional Reactive programs}
Most of the typing for financial DSLs is based on \emph{functional reactive programming}, i.e.,
pure functions over signals. This includes loops as well.

Linear-time Temporal Logic (LTL) cite:Jeffrey:2012fh, cite:Pnueli:1977wy is a
natural extension of the type system for FRP, which constrains the temporal
behaviour of reactive programs.

LTL can be defined in a dependently typed functional language, and that
reactive programs form proofs of constructive LTL properties. Types are indexed over time:

$$RSet = Time \rightarrow Set$$
\end{frame}





\begin{frame}[label={sec:org999fdee}]{and temporal logic cite:Pnueli:1977wy}
Correctness over time can be reduced to

\begin{itemize}
\item \alert{Invariance}: a property holding throughout the execution of a program

\item \alert{Eventuality}: temporal implication, or a property \(Q\) eventually follows from \(P\)
\end{itemize}
\end{frame}










\section{References}
\label{sec:org5f822b9}
\begin{frame}[label={sec:orga4b2e3d}]{Papers}
bibliographystyle:unsrt
bibliography:biblio.bib
\end{frame}
\end{document}
